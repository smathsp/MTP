\chapter{数学教学中的认识信念}

\section{认识信念的概念}


\subsection{个体的认识信念}
个体的认识信念是个体对知识和知识认识过程的朴素看法或观点。

个体认识信念的基本研究致力于回答两个基本问题:
\begin{enumerate}
    \item 个体或学生对知识本质的认识是什么样的。
    \item 个体或学生对知识是如何获得的认识,即\textbf{认识过程}的认识是怎样的。
\end{enumerate}

\section{数学教学观}
\begin{itemize}
    \item \textbf{绝对主义:} 数学是确定无误的真理,永恒、至高无上(静态)。
    \item \textbf{可误主义:} 数学理论不是绝对确定的(可误),数学理论会不断发展(动态)。
\end{itemize}

\begin{multicols}{2}
\begin{itemize}
    \item \textbf{绝对主义的数学教学观}
    \begin{enumerate}
        \item \textbf{静态}知识是种结果,“结果型”教学模式。
        \item 数学过程存在于知识接受。
    \end{enumerate}
    \item 教学方式:
    \begin{enumerate}
        \item 归纳的方式展示教学内容。
        \item 强调归纳的思维训练。
        \item 偏向论证,追求数学的完备性。
        \item 偏证实(接受学习)。
    \end{enumerate}
\end{itemize}
\columnbreak
\begin{itemize}
    \item \textbf{可谬主义的数学教学观}
    \begin{enumerate}
        \item \textbf{动态}知识是种过程,“过程型”教学模式。
        \item 教学过程充分展示知识的发生与发展过程,通过发现问题、解决问题的过程去获取知识。
    \end{enumerate}
    \item 教学方式:
    \begin{enumerate}
        \item 演绎形式展示内容。
        \item 强调演绎思维训练。
        \item 偏向实验。
        \item 偏向证伪(研究学习)。
    \end{enumerate}
\end{itemize}
\end{multicols}

\textbf{合理的数学观:}
    \begin{enumerate}
        \item 从认识论层面看,数学知识不是真理,是可误的,是动态发展的。
        \item 从方法论层面看,数学是归纳与演绎、实验与论证、证实与证伪的统一。
    \end{enumerate}

\section{数学观对数学教学观的影响}

\begin{multicols}{2}
\begin{itemize}
    \item \textbf{静态的数学观:}
    \begin{enumerate}
        \item 数学过程是数学知识接受过程。
        \item 数学模式是一种“结果型”范型。
        \item 数学评价偏重于掌握知识数量的检测。
    \end{enumerate}
\end{itemize}

\columnbreak

\begin{itemize}
    \item \textbf{动态的数学观:}
    \begin{enumerate}
        \item 数学过程是充分展示知识的发生、发展过程。
        \item 数学模式是一种“过程型”范型。
        \item 数学评价包含知识掌握、思维训练和能力发展的考查。
    \end{enumerate}
\end{itemize}

\end{multicols}
\clearpage
\begin{multicols}{2}
\begin{itemize}
    \item \textbf{方法论层面的数学观:}
    \begin{itemize}
        \item 归纳与演绎。
        \item 实验与论证。
        \item 证实与证伪。
    \end{itemize}
    \item \textbf{理性主义偏向的数学观:}
    \begin{enumerate}
        \item 强调数学结构、概念层次和严密性。
        \item 学生通过数学学习领略数学的内在价值,欣赏数学美。
        \item 教师的作用在于有意义的讲授、解释,并传递数学结构。
    \end{enumerate}
\end{itemize}
\columnbreak
\begin{itemize}
    \item \textbf{数学教学观:}
    \begin{itemize}
        \item 教学中的归纳取向与演绎取向
        \item 教学中的实验取向与论证取向
        \item 教学中强调证明还是反驳。
    \end{itemize}
    \item \textbf{现实主义偏向的数学观:}
    \begin{enumerate}
        \item 注重知识的工具性。
        \item 使学生获得就业需要的数学知识和技能,能使用数学去解决现实生活和生产中的问题。
        \item 着重选择有应用背景的知识作为教学材料。
        \item 注重实践性活动。
    \end{enumerate}
\end{itemize}
\end{multicols}


\section{认识信念对数学学习的影响}

\section{数学教学认识信念}



\section{非智力因素对数学学习的影响:}

    \begin{itemize}
        \item \textbf{动机因素:}
        \begin{enumerate}
            \item \textbf{期望-价值因素:} 行为的发生依赖于人们对导致目标实现性的认识,依存于目标的主观价值。
            \item \textbf{成功归因因素:} 确定成功与失败的原因。
            \item \textbf{能力及自我知觉因素:} 包括自我价值和自我效能感。
            \item \textbf{学习目标因素:} 学习目标与成绩目标。
        \end{enumerate}
        \item \textbf{焦虑因素:} 个体预料会有某种不良后果或模糊性威胁将出现时产生的一种紧张不安情绪状态。
        \item \textbf{自我效能感:} 个体对自己实现特定领域行为目标所需能力的信心或信念。
    \end{itemize}

