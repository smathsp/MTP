\chapter{数学学习策略}


\textbf{丹塞克:学习策略}
\begin{itemize}
    \item \textbf{基础策略}:直接操作材料的各种学习策略,主要包括信息获取、贮存、信息检索和应用的策略。
    \item \textbf{支持策略}:帮助学习者维持适当的认知氛围,以保证基础策略有效运作的策略。包括计划和时间筹划,注意力分配与自我监控和诊断策略。
\end{itemize}


% 表8-1
\begin{table}[h]
    \centering
    \caption{学习策略分类}
    \begin{tabular}{m{3cm}m{3cm}m{6cm}}
        \toprule
        \textbf{学习策略类别} & \textbf{策略} & \textbf{具体策略} \\
        \midrule
        \multirow{3}{*}{认知策略} 
        & 复述策略 & 如复习、抄写、作记录、画笔记 \\
        & 精加工策略 & 如想象、口述、总结、做笔记、类比、答题等 \\
        & 组织策略 & 如组块、选择要点、列提纲、画地图等 \\
        \midrule
        \multirow{3}{*}{元认知策略}
        & 计划策略 & 如设置目标、浏览、设疑等 \\
        & 监视策略 & 如自我监控、集中注意、回想视频会等 \\
        & 调节策略 & 如调整阅读速度、重新阅读、复查、使用应试策略等 \\
        \midrule
        \multirow{4}{*}{资源管理策略}
        & 时间管理 & 如建立时间表、设置目标等 \\
        & 学习环境管理 & 如寻找固定地方、安排地方、有组织的地方等 \\
        & 努力管理 & 如归因于努力、调整心境、自我谈话、坚持不懈等 \\
        & 他人的支持 & 如寻求教师、伙伴帮助、获得个别指导等 \\
        \bottomrule
    \end{tabular}
\end{table}

% 表8-2
\begin{table}[h]
    \centering
    \caption{常用的学习策略}
    \begin{tabular}{m{3cm}m{8cm}}
        \toprule
        \textbf{策略} & \textbf{功能} \\
        \midrule
        提问 & 确定假设,建立目标及任务参量,寻求反馈,将当前任务与先前经验联系 \\
        计划 & 制订策略及实施计划表,将任务或问题分解,确定必需的动作步骤及技能 \\
        监控 & 不断检查计划行为目标是否匹配,回答发现目标问题原因或目标 \\
        检查 & 对执行过程和结果作进一步评价 \\
        修正 & 进行重新设计或改良计划,或对新目标进行设定 \\
        自我评价 & 对任务的执行过程和结果进行最终的自我评价 \\
        \bottomrule
    \end{tabular}
\end{table}