\chapter{数学学习迁移}

\section{迁移的概念}

迁移——一种学习对另一种学习的影响

先前学习$\xrightarrow{\text{顺向迁移}}$后继学习

先前学习$\xleftarrow{\text{逆向迁移}}$后继学习

正迁移——积极的促进作用

负迁移——消极影响或阻碍作用

\section{迁移理论}

\subsection{早期迁移理论}

\begin{enumerate}
    \item \textbf{形式训练说}:官能心理学。
    \item \textbf{相同要素说}:行为主义心理学(桑代克)。
    \item \textbf{概括说}:贾德。$A \to B$ 是因为有相同的原理 $C$ (“水下击靶”实验:A 组练习游泳击靶,B 组不练习)。
    \item \textbf{关系转化说}:格式塔心理学,强调学习过程中整体结构的形成和转化。(苛勒的试验)
\end{enumerate}



\subsection{现代迁移理论}


\begin{enumerate}
    \item \textbf{认知结构迁移理论} —— 认知心理学(奥苏贝尔、布鲁纳、罗耶)\\
    影响迁移的不是经验的一组刺激-反应的联结,而是认知结构。\\
    影响迁移的不是前后两个学习课题在刺激和反应方面的相似程度,而是认知结构的组织和联系。
    
    \item \textbf{产生式迁移理论}(安德森)\\
    基本思想:先后两项技能学习产生迁移的原因是两项技能之间产生式的重叠。重叠越多,迁移量越大。
    
    \item \textbf{类比迁移理论}\\
    概念:当人们遇到一个新问题(靶问题),往往想起一个过去已经解决的相似问题(源问题),并运用源问题的解决方法和程序去解决靶问题。\\
    类比迁移理论分为:
    \begin{enumerate}
        \item 结构映射理论:类比迁移是结构映射过程,源问题各因素间的关系,被提取并被作用于靶问题。
        \item 实用图式理论。
        \item 示例理论。
    \end{enumerate}
\end{enumerate}


\section{样例}

样例是一种教学手段,它给学习者提供了专家的问题解决方法,以供其研习模仿。

样例包含以下三个部分:
\begin{itemize}
    \item \textbf{问题}:对要求学生解决的问题作了陈述。
    \item \textbf{解决问题的方法}:逐步描述了解决问题的步骤。
    \item \textbf{评论}:解释了采取每一步的理由或根据。
\end{itemize}
\clearpage
\subsection*{样例的设计}
\begin{multicols}{2}
\begin{enumerate}
    \item \textbf{子目标学习模型}:通过附着框架或用视觉分离的方法,清晰地表示出:
    \begin{enumerate}
        \item 子目标。
        \item 实现子目标所需的策略或方法。
        \item 子目标间的等级关系。
    \end{enumerate}
    
    \item \textbf{自我解释效应}:在样例中穿插一系列反省问题,引发学习者自我解释。\\
    自我解释的焦点在于理解学习材料和搞清楚学习材料的意图。
\columnbreak
    \item \textbf{条件建构—优化理论}:为建构和优化产生式规则的条件,应包含:
    \begin{enumerate}
        \item 提供产生式规则的有解例题。
        \item 成组安排的例题。
        \item 用言语陈述所学规则的小结。
    \end{enumerate}

    \item \textbf{多重样例理论}:要求注意:
    \begin{enumerate}
        \item 多重样例的数量。
        \item 样例间的变异性。
        \item 多重样例的呈现方式:
        \begin{itemize}
            \item 交互式。
            \item 分块式。
            \item 不完整式。
        \end{itemize}
    \end{enumerate}
\end{enumerate}
\end{multicols}

\subsection{样例对迁移的影响}
样例所包含的信息分为:
\begin{itemize}
    \item \textbf{内在原理信息}:问题所包含的内在结构或关系。
    \item \textbf{表面内容信息}:问题所涉及的事物、形式、情节等具体内容。
\end{itemize}

\begin{center}
当前要解决的问题(靶题)\\
(原理通达)$\uparrow$$\downarrow$(原理运用)\\
之前学过的样例(源题)
\end{center}

\subsubsection*{源题—靶题的迁移过程(雷利约克)}
\begin{enumerate}
    \item \textbf{原理通达(类比源的选取)}:搜索记忆中可供参考的样例,以确定新问题应用哪个原理去解决。
    \item \textbf{原理运用(关系的匹配)}:新问题与样例的各个部分进行匹配,根据匹配结果产生解决问题的方法。
\end{enumerate}

样例影响通达(类比源的选取)和运用(关系的匹配)


类比学习的过程包括:
\begin{enumerate}
    \item 信息输入
    \item 模式匹配
    \item 核验
    \item 修正
\end{enumerate}


\begin{tabular}{c|cc}
        & 结构相似 & 结构差异  \\
\hline
表面相似 &   &    \\
表面差异 &       &      \\
\end{tabular}

\clearpage

\section{促进正迁移的教学策略}

\subsection{提高学生数学概括能力}

\begin{multicols}{2}
    


\subsubsection*{Q1: 在数学概念学习中如何提高概括能力?}

数学概念学习有两种方式:
\begin{itemize}
    \item \textbf{概念同化}。
    \item \textbf{概念形成}:在观察、分析、比较、归纳的基础上进行概括。
\end{itemize}

\textbf{方法:}
\begin{enumerate}
    \item 创设适合学生认知概念的问题情境。
    \item 采用恰当的提问方式展开教学。
    \item 教师对学生的观点,给出积极评价。
\end{enumerate}

\subsubsection*{Q2. 解题练习中如何提高概括能力?}

\textbf{解决问题的概括:}
\begin{itemize}
    \item 对知识的概括。
    \item 对方法的概括。
\end{itemize}

\columnbreak

\subsubsection*{Q3. 在数学知识复习中如何提高概括能力?}
\textbf{数学知识复习:}
\begin{enumerate}
    \item 梳理知识结构。
    \item 提取数学思维方法。
    \item 加强数学知识的应用。
\end{enumerate}

\textbf{对应策略:}
\begin{enumerate}
    \item 根据知识间的逻辑关系创建知识结构图,分析内在关系。
    \item 剖析隐含在知识中的数学思想方法。
    \item 对知识图进行拓广,与先前学习的章节建立联系,形成概念域、概念系、命题域、命题系。
    \item 对各理论、公式、法则的条件、使用程序、使用情境的总结与归纳。
\end{enumerate}

\end{multicols}

\begin{multicols}{2}
\subsection{帮助学生建立完善的数学认知结构}
\begin{enumerate}
    \item 促进学生对陈述性知识的组织和精度加深。
    \item 促进学生图式的形成。
    \item 促进学生产生式系统的建构。
\end{enumerate}


\subsection{训练类比推理能力 —— 基于类比迁移理论}
\begin{enumerate}
    \item 类比源多样化与类比多次化。
    \item 从明确指导到自我监控的过渡。
\end{enumerate}

\columnbreak

\subsection{选择恰当样例组织教学}
\begin{enumerate}
    \item 样例呈现的多重性。
    \item 样例中设置子目标。
    \item 样例的变式。
    \item 示例演练教学。
    \item 正确处理不同样例数学习题的关系。
\end{enumerate}
\end{multicols}