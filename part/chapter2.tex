\chapter{数学认知结构}

\section{认知结构的内涵}

数学教学的一个基本任务在于帮助学生形成良好的\textcolor{red}{数学认知结构},以满足后继学习的需要,最终提高学生的问题解决能力。

\begin{multicols}{2}

\subsection{皮亚杰的图式理论}

定义:图式是指个体对世界的知觉、理解和思考方式,可以视为心理活动的框架或组织结构。

\begin{itemize}
    \item 图式是认知结构的核心。
    \item 图式的发展是人类认识事物的基础,表现为图式的形成和发展过程。
\end{itemize}

\vspace{\baselineskip}


图式发展的三个过程:
\begin{itemize}
    \item 同化:
    \begin{itemize}
        \item 个体将外部信息纳入已有图式的过程。
        \item 是在已有图式的基础上形成新的图式。
    \end{itemize}
    \item 顺应:
    \begin{itemize}
        \item 个体调整内部结构以适应外部刺激的过程。
        \item 当遇到无法用已有图式解释的新刺激时,需要对原有图式修正、改组或重建。
    \end{itemize}
    \item 平衡:
    \begin{itemize}
        \item 个体通过自我调节使认知发展从一个平衡状态过渡到另一个平衡状态。
        \item 若原有图式不足以适应新刺激,则通过“顺应”调整达到新的平衡。
    \end{itemize}
\end{itemize}

\columnbreak

\subsection{奥苏伯尔的认知结构}

认知结构:
\begin{itemize}
    \item 指某一人的各种观念的全部内容与组织形式。
    \item 在教学或学习中,指学生在特定知识领域内的各种观念的组织。
\end{itemize}

\vspace{\baselineskip}

对认知结构的理解:
\begin{itemize}
    \item 认知结构的解释包含两个层次:
    \begin{itemize}
        \item 广义:学习者观念的全部内容与组织结构。
        \item 狭义:特定领域内的知识组织方式。
    \end{itemize}
    \item 三个变量:
    \begin{itemize}
        \item 可利用性:原认知结构中相关知识能否用以理解新概念。
        \item 可辨性:新旧概念之间的差异是否明确,防止混淆。
        \item 稳定性:原认知结构对新知识的固定作用,影响是否能形成稳定的知识点。
    \end{itemize}
\end{itemize}

\end{multicols}

\subsection{知识结构与认知结构的异同}

\begin{itemize}
    \item 内涵不同
    \begin{itemize}
        \item 知识结构:以外显的文本形式表现的知识体系;
        \item 认知结构:经过学习者主观改造的知识结构,
    \end{itemize}
    \item 结构的构造方式不同
    \begin{itemize}
        \item 知识结构:相对严密的逻辑体系,结构完善
        \item 认知结构:个人建构,可能出现残缺、不完整、曲解等
    \end{itemize}
    \item 内容的完备性不同
    \begin{itemize}
        \item 知识结构:相对系统、完备、无缺口,涵盖它的全部组成内容 
        \item 认知结构:由于接受、理解和遗忘等原因,往往是不完备的
    \end{itemize}
\end{itemize}

\section{数学概念}

\textbf{三个特征:}

\begin{itemize}
    \item 对同一个概念,可以从不同的侧面或角度去刻画,即可以采用彼此等价的一组定义去描述同一概念。
    \item 概念具有发展性,在不同的背景下可以赋予一个概念新的意义。
    \item 数学概念不是孤立的,定义一个新概念往往用到诸多旧概念。
\end{itemize}

\subsection{概念域}

图式是对同类事物的命题或知觉的共性的编码方式。

一个概念 $C$ 的所有等价定义的公式,叫作概念 $C$ 的概念域。

\textbf{概念域的含义:}——由同一数学概念的不同等价定义所构成。

\begin{itemize}
    \item 概念域是个体对数学概念的一种心理表征。
    \item 概念域是某个概念的一些等价定义在个体头脑中形成的命题网络和表象。
    \item 命题网络中各节点的关系是等价关系。
    \item 在概念域的命题网络中存在一个典型例题。
\end{itemize}

\textbf{广义概念域:}具有同构关系的概念网络的图式。

\subsection{概念系}

概念系由强、弱、广义抽象关系的数学概念所构成的命题网络,包含概念域。

\begin{enumerate}
    \item \textbf{弱抽象}:对某一特征抽象。
    \item \textbf{强抽象}:引入新属性。
    \item \textbf{广义抽象}:用到了(在定义 $B$ 用到了$A$ 时,$B$ 是 $A$ 的广义抽象)。
\end{enumerate}

\subsection{命题域与命题系}

\begin{enumerate}
    \item \textbf{命题域}:等价关系。由一个数学真命题的不同等价形式所构成。
    \item \textbf{命题系}:向含等价、广义抽象、产生式系统关系。\\
    指具有等价关系、广义抽象关系的数学真命题所构成的产生式系统关系,它包含了命题域。
\end{enumerate}


\section{CPFS 结构}

\begin{multicols}{2}

概念域(CF)、概念系(CS)、命题域(PF)、命题系(PS)形成的结构称为 \textbf{CPFS 结构}。

\textbf{CPFS 结构与数学认知结构的关系}
\begin{enumerate}
    \item CPFS 结构是数学学习特有的认知结构。
    \item CPFS 结构是优良的数学认知结构。
\end{enumerate}

\columnbreak

数学学习的一个基本任务是形成良好的认知结构。

奥苏贝尔用 3 个认知结构变量去评判优劣:
\begin{enumerate}
    \item 可利用性
    \item 可辨别性
    \item 稳定性
\end{enumerate}

\end{multicols}

\clearpage

\section{概念图}

\subsection{什么是概念图?}  
概念图是一种用来组织和表征知识的实用工具,是一种以科学命题的形式显示概念间的意义联系,并用具体事例加以说明,从而把所有的基本概念有机联系起来的空间网络结构图。

\subsection{概念图的构造}
\begin{enumerate}
    \item 把某一单元所有的概念罗列出来。
    \item 找出这些概念中较为基本的带有普遍意义的关键概念。
    \item 从关键概念出发,寻求各概念之间的联系,然后按一定的逻辑关系将所有的概念整理归类。
    \item 建立概念间的连结,并在连线上用联结词标注两者之间的关系。
    \item 不断反思和完善概念图。
\end{enumerate}

\subsection{在小学数学教学中,如何运用概念图教学?}
\begin{enumerate}
    \item 渗透到每一节课的小节结中。
    \item 随着概念的增加,逐步扩展。
    \item 重视学生对概念间关系的识别。
\end{enumerate}
