\chapter{数学学习中的元认知因素}

\section{元认知}

\subsection*{元认知结构}
元认知包括:
\begin{itemize}
    \item \textbf{元认知知识}:个体关于自己或他人的认知活动、过程、结果影响因素等方面的知识。
    \item \textbf{元认知体验}:个体从事认知活动时产生的认知情感体验。
    \item \textbf{元认知监控}:个体对自己正在进行的活动进行积极、自觉的监控、控制和调节。
\end{itemize}

\begin{multicols}{2}

\subsection*{元认知知识分类}
\begin{enumerate}
    \item \textbf{关于任务和目标的认知}:
    \begin{itemize}
        \item 对认知材料的认知。
        \item 对任务性质的认知。
    \end{itemize}
    \item \textbf{关于策略的认知}。
    \item \textbf{关于个体的认知}:
    \begin{itemize}
        \item 个体内部差异的认知。
        \item 个体间差异的认知。
        \item 对自我认知发展认知。
    \end{itemize}
\end{enumerate}

\columnbreak

\subsection*{元认知监控的过程}
\begin{enumerate}
    \item 制订计划。
    \item 执行控制。
    \item 检查结果。
    \item 采取补救措施。
\end{enumerate}

\end{multicols}

\subsection*{元认知的特点}
\begin{enumerate}
    \item \textbf{自我意识性}:\\
    自我意识以主体自身及其活动为意识对象,因而对人的认知活动起监控作用。
    
    \item \textbf{能动性}:\\
    元认知完全是主体独立地、自觉地认识、体验和调节自己的认知活动。
    
    \item \textbf{调节性}:\\
    元认知核心问题是调节认知活动。无论是认知目标的确定、认知计划的制订和执行、认知策略的选择,还是认知时间、内容、程序的安排、认知效果评价等,都是主体对自己认知系统相关因素的调节活动。
    
    \item \textbf{反馈性}:\\
    主体不断获取认知活动系统中各要素变化的信息,检测和评价认知活动的过程及效果。
    
    \item \textbf{迁移性}:\\
    主体从某情境某学科某领域习得的自觉地评价、计划、总结、调节、修正,补救等技能,不仅能积极的迁移到本学科、本领域或其他方面,而且可自觉地适用于另一情境、学科、领域,从而对学习产生影响。(举一反三,触类旁通)
\end{enumerate}

\clearpage

\section{元认知与认知的区别}

\begin{enumerate}
    \item \textbf{内容方面}:
    \begin{itemize}
        \item 认知:对客体进行某种智力操作。
        \item 元认知:对认知主体正在进行的认知活动过程及结果进行智力操作。
    \end{itemize}
    
    \item \textbf{对象方面}:
    \begin{itemize}
        \item 认知:对象是外在的具体事物。
        \item 元认知:对象是主体自身(作为一个独立的客体)。
    \end{itemize}
    
    \item \textbf{目的方面}:
    \begin{itemize}
        \item 认知:目的是认知主体取得认知活动的具体结果。
        \item 元认知:目的是监控认知活动的发展,给主体提供有关进展的信息,间接地促进和推动这种进展。
    \end{itemize}
\end{enumerate}

\noindent 两者最终目的是一致的——使主体完成认知任务,实现认知目标。



\section{学生自我监控能力发展规律}

\textbf{研究者}:章建跃

\textbf{研究方法}:问卷调查和测试

\subsection*{研究结果}
\begin{enumerate}
    \item 在正常的学校教育条件下,中学生数学学科自我监控能力的发展有其年龄阶段性,但发展速度比较平缓。
    \item 中学生数学学科自我监控能力的发展具有其规律性:
    \begin{itemize}
        \item 他控$\xrightarrow{}$自控;不自觉$\xrightarrow{}$自动化;单维$\xrightarrow{}$多维;局部$\xrightarrow{}$整体;敏感性$\uparrow$;迁移性$\uparrow$。
    \end{itemize}
    \item 在中学阶段,不同年级的学生的自我监控能力没有显著性差异。
    \item 中学生数学学科自我监控能力的发展落后于其它心理能力的发展。
\end{enumerate}

\begin{multicols}{2}
\subsection*{他控 $\rightarrow$ 自控}

\begin{enumerate}
    \item 初中生更多依赖于教师的控制,自控能力较弱。
    \item 随着年龄增长,由于个体生理的成熟、心理的发展、个体知识的积累和实践经验的丰富,自我监控能力会逐步提高。
\end{enumerate}

\subsection*{不自觉$\rightarrow$自动化}

初中低年级学生在数学学习中的自我监控不自觉,表现在:

\begin{itemize}
    \item 学习中缺乏计划性,盲目尝试的成分大;
    \item 缺乏必要的检验技能;
    \item 解题的逻辑性不强,缺乏系统性、条理性。
\end{itemize}
\columnbreak
\subsection*{单维 $\rightarrow$ 多维}

\textbf{1. 国外有关研究发现,为了提高学业成绩}
\begin{itemize}
    \item 小学生采取的主要方法就是增加学习时间,加大学习强度;
    \item 中学生的做法则是,根据个人具体特点,改进学习方法,调节努力程度等。
\end{itemize}

\textbf{2. 进一步,董奇等人的研究表明}
\begin{itemize}
    \item 低年级学生多从学习和认知过程、具体学习方法或学习目标任务进行监控与调节;
    \item 忽视非认知因素中努力程度、动机激发、性格特征、认知风格等多方面的监控与调节。
\end{itemize}
\end{multicols}
\clearpage
\subsection*{局部 $\rightarrow$ 整体}

\textbf{1. 低年级学生更多对数学学习过程进行反馈和矫正}
他们只对学习的结果进行检查,缺乏对学习过程的评估意识。

\textbf{2. 高年级学生逐步形成了对学习活动各阶段的自我监控意识}
\begin{itemize}
    \item 例如:活动起始阶段的目标确立、方法选择;
    \item 活动实施阶段的反馈、调节与控制执行;
    \item 活动结束阶段的效果评估与补救修正等。
\end{itemize}

\subsection*{敏感性$\uparrow$}

\begin{enumerate}
    \item 自我监控的敏感性
    \item 高年级学生在数学学习中两种敏感性更高
    \begin{itemize}
        \item 问题情境中寻找线索
        \item 不同问题情境选择恰当解决策略,知识经验的激活和提取
    \end{itemize}
\end{enumerate}

\subsection*{迁移性$\uparrow$}

\begin{enumerate}
    \item 中学生的自我监控水平与他们所掌握的数学思想方法密切相关。
    \item 数学思想方法的抽象水平越高,其自我监控的迁移水平越高。
\end{enumerate}